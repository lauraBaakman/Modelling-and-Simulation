%!TEX root = practicum2.tex
\todo[inline]{Hoe verandert de fractal dimension $\rho$ als een functie van $p$?}

\textcite{falconer2004fractal} describes the fractal dimension as some number $\rho$ such that
\begin{equation}
	M_\varepsilon(\rho) \sim c\varepsilon^{-s}
\end{equation}
where $c$ and $s$ are constants and $M_\varepsilon(\rho)$ are measurements at different scales $\varepsilon$ for $\varepsilon \to 0$. \citeauthor{falconer2004fractal} then shows that the fractal dimension can be estimated ``as minus the gradient of a log-log graph plotted over a suitable range of $\varepsilon$"\cite{falconer2004fractal}. 

One way to get the measurements $M_\varepsilon$ is to use box-counting. When box-counting is used the different scales, mentioned in \citeauthor{falconer2004fractal}'s definition, are the sizes of the boxes.

We have used the function \t{box-count} by \textcite{boxCounting} to determine the fractal dimension of Percolation clusters. This method uses box sizes that are power of two. Consequently $\varepsilon = 1, 2, 4, \dotsc 2^q$ where $q$ is the smallest integer such that $q \leq (2N + 1)$. 

We have used the box-counting algorithm on a cluster generated with $N = 80$, $p = 0.7$, the used cluster is shown in \cref{fig:exp_fractal:cluster}. \Cref{fig:exp_fractal:fractalDimension} presents the number of boxes as a function of the size of the boxes. The box-counting dimension can be read from \cref{fig:exp_fractal:fractalDimensionGradient} to be \num{1.879}, which neatly approximates the dimension \num{1.896} mentioned by \textcite{stauffer1994introduction}. \todo[inline]{Uileggen hoe we dimension uit dit figuur halen.} The small difference between these numbers can be explained by the relatively small size of our cluster. 

dfs

\begin{figure}
	\centering
	\includegraphics[width=\columnwidth]{./img/assignment_fractal_rhoVSp}
	\caption{The average box-counting dimension of 20 finite clusters for each value of $p$ for $p = 0.3, 0.4, \dotsc, 0.7$.}
	\label{fig:assignment:fractal:dimensionVSp}
\end{figure}

