\documentclass[twoside, a4paper, twocolumn, leqno, fleqn, 11pt]{article}
\usepackage[english]{babel}
\usepackage{a4wide}

% Fonts
\usepackage[sc]{mathpazo}
\usepackage[T1]{fontenc}
\linespread{1.05}
\usepackage{lmodern}
\usepackage{microtype}
\usepackage{lettrine}

% Document layout
% \usepackage[hmarginratio=1:1,top=32mm,columnsep=20pt]{geometry}
\usepackage{paralist}

% Floats
\usepackage{float}

%Plaatjes
\usepackage{tikz}
\usepackage{graphicx}
\usepackage[hang, small,labelfont=bf,up,textfont=sl,up, justification=raggedright]{caption}
\usepackage{subcaption}
\DeclareCaptionLabelFormat{opening}{(#2)}
\captionsetup{subrefformat=opening}

% Tables
\usepackage{booktabs}

% Custom section headers
\usepackage{titlesec}
\titleformat{\section}[block]{\large\scshape}{\thesection.}{1em}{}
\titleformat{\subsection}[block]{\large\scshape}{\thesubsection.}{1em}{}
\titleformat{\subsubsection}[block]{\scshape}{\thesubsubsection.}{1em}{}

%Wiskunde
% \usepackage[scientific-notation=true, round-mode=figures, round-precision=4, exponent-base=e, exponent-product={}]{siunitx}
\usepackage[low=1e-1,high=1e3]{threshold}
\sisetup{round-mode=figures, round-precision=4, exponent-base=e, exponent-product={}}
\usepackage{mathtools}
\usepackage{amsmath}
\usepackage{amsfonts}
\usepackage{amssymb}

% References
\usepackage{varioref}
\usepackage{hyperref}
\usepackage[noabbrev]{cleveref}

% References
\usepackage[backend=bibtex]{biblatex}
\usepackage{csquotes}
\bibliography{biblio}

% Pseudo code
\usepackage[algoruled,vlined, linesnumbered, shortend]{algorithm2e}
\usepackage[export]{adjustbox}
\usepackage{setspace}

% Temp
\usepackage[obeyFinal]{todonotes}
% \usepackage{showlabels}

% Math commands
\newcommand{\normal}[2]{\ensuremath{\mathcal{N}\left(#1,\, #2\right)}}
\renewcommand{\vec}[1]{\ensuremath{\mathbf{#1}}}

% % Lists
% \usepackage{paralist}

\crefname{algocf}{algorithm}{algorithms}
\Crefname{algocf}{Algorithm}{Algorithms}
\crefname{algocfline}{line}{lines}
\Crefname{algocfline}{Line}{Lines}
% Other commands
\renewcommand{\t}[1]{\texttt{#1}}

% \usepackage{float}
 %    \renewcommand{\topfraction}{0.9}	% max fraction of floats at top
 %    \renewcommand{\bottomfraction}{0.8}	% max fraction of floats at bottom
 %    %   Parameters for TEXT pages (not float pages):
 %    \setcounter{topnumber}{2}
 %    \setcounter{bottomnumber}{2}
 %    \setcounter{totalnumber}{4}     % 2 may work better
 %    \setcounter{dbltopnumber}{2}    % for 2-column pages
 %    \renewcommand{\dbltopfraction}{0.9}	% fit big float above 2-col. text
 %    \renewcommand{\textfraction}{0.01}	% allow minimal text w. figs
 %    %   Parameters for FLOAT pages (not text pages):
 %    \renewcommand{\floatpagefraction}{0.1}	% require fuller float pages
	% % N.B.: floatpagefraction MUST be less than topfraction !!
 %    \renewcommand{\dblfloatpagefraction}{0.7}	% require fuller float pages

\title{Modelling and Simulation\\Practical Assignment 2: Percolation}
\author{%
	Rick van Veen (s1883933)%
	\thanks{These authors contributed equally to this work.}% 
	\and% 
	Laura Baakman (s1869140)%
	\footnotemark[1]%
}


\begin{document}

\maketitle

% \todo[inline]{Use compactitem environment for lists}

% \section{Introduction}
% %!TEX root = practicum1.tex
\todo[inline]{Mogelijk uitbreiding als Han er zo niet gelukkig mee is: bespreek Winding numbers, zie \cite{kenzel1997physics}, en bepsreek de praktische toepassing van de map in de inleiding.}

This paper discusses the Chirikov map, since we are only interested in the non-integer part of the Chirikov map we the following modified map:
\begin{subequations}\label{eq:chirikov}
	\begin{align}
		\label{eq:chirikov:p} p_{n + 1} &= p_n + \frac{K \sin \left(  2 \pi x_n \right)}{2 \pi} \mod 1 \\
		\label{eq:chirikov:x} x_{n + 1} &= x_n + p_{n + 1} \mod 1,
	\end{align}
\end{subequations}	
where the non-linearity parameter $K \in \mathbb{R}$. The modulo operator ensures that $x_n, p_n \in \left[ 0, 1 \right]$. In this paper we discuss the influence of the different parameters on the orbits in the map.

% \section{Method}
% \label{s:method}
% %!TEX root = practicum2.tex
\todo[inline]{Inleiding in experiment}
\todo[inline]{Pseudo code}
\todo[inline]{Iets over de exacte stopconditie}	

\section{Experiments}
\label{s:experiment}
%!TEX root = practicum2.tex
\todo[inline]{Inleiding in experiment}


\subsection{Probability}
\label{ss:exp:probability}
%!TEX root = practicum2.tex
\begin{figure}
	\centering
	\includegraphics[width=\columnwidth]{./img/assignment_a_p_infinite_ratio_p.pdf}
	\caption{Ratio of percolating clusters, $P_\infty$, as a function of $p$. Ratios are calculated over $r_{max} = 200$ runs on a $41 \times 41$ grid.}
	\label{fig:experiment:prob:p_inf_ratio}
\end{figure}


\begin{figure*}
	\centering
	\includegraphics[width=\textwidth]{./img/assignment_a_mean_std_p.pdf}
	\caption{Mean cluster sizes, represented as points, and standard deviations, indicated by the vertical error bars, as a function of $p$, with a step size of $0.1$. The mean and standard deviation were calculated over $200$ runs on a $41 \times 41$ grid.}
	\label{fig:experiment:prob:mean_std_clusters}
\end{figure*}

% Theory
One important property of clusters is their size, and how that size depends on the parameter $p$. Since one can only determine the size of a finite cluster, we only consider size to be defined for non-percolating clusters. Therefore in the following discussion on the size of clusters we do not consider percolation clusters. \textcite{kenzel1997physics} describe this relation as follows: for small values of $p$ we get a large number of small clusters. As $p$ increases we find positive correlation between $p$ and the average cluster size until $p$ reaches some threshold value $p_c$. For $p > p_c$ we get either a small finite cluster or a percolating cluster. As $p > p_c$ increases the probability of ending with a finite cluster decreases, until we always get the percolating cluster for $p =1 $. Note that although in theory this cluster should cover the full grid, this is not necessarily the case in our model, since it stops growing as soon as one border site is occupied. \\

% Ons experiment
To find an indication of the value of $p_c$ with our model we have let it generate a cluster on a $41 \times 41$ grid for $p = 0.31, 0.32, \dotsc, 0.7$. For each value of $p$ we grow $r_{max} = 200$ clusters. 

\Cref{fig:experiment:prob:mean_std_clusters} presents the mean and standard deviation of the size of the finite clusters as a function of $p$. In this figure we observe the effect of $p$ on the mean cluster size described by \citeauthor{kenzel1997physics}. Furthermore, based on these data one would estimate $p_c$ to be approximately $0.55$. 

\citeauthor{kenzel1997physics} also predicted that the number of percolating clusters relative to the number of finite clusters would grow for $p > p_c$ until $p = 1$, where the only possibility would be a percolating cluster. To observe this effect \cref{fig:experiment:prob:p_inf_ratio} shows $P_\infty$, which is the ratio of the number of percolating clusters to the number of finite clusters. This graph is based on the same data as \cref{fig:experiment:prob:mean_std_clusters}. Based on this graph we would say that $p_c \approx 0.4$. This number is lower than the value for $p_c$ based on \cref{fig:experiment:prob:mean_std_clusters}. This is probably caused by the relatively small grid sizes, which causes us to classify some clusters as percolating, that are actually finite. 

\textcite{stauffer1994introduction} has found $p_c$ to be approximately \num{0.59275} for a square lattice. As stated earlier our lower estimation of $p_c$ is quite likely caused by our small grid. 


\subsection{System Size}
\label{ss:exp:systemSize}
%!TEX root = practicum2.tex
	\todo[inline]{How do the results change when the system size changes. Experiment with different latice sizes}
	\todo[inline]{Wat could the behavior be in the limit of infinite lattice sizes}

\subsection{Fractal Dimension}
\label{ss:exp:fractal}
%!TEX root = practicum2.tex
% \todo[inline]{Bonus: Determine the fractal dimension of finite clusters as a function of $p$.}
\todo[inline]{Definitie van fractal dimension uit een reputable source trekken, huidige is van wikipedia}
\textcite{} defines a fractal dimension as a ratio providing a statistical index of complexity comparing how detail in a pattern changes with the scale at which it is measured. 

\todo[inline]{BoxCounting uitleggen}

We have used the function \t{boxcount} by \textcite{boxCounting} to determine the fractal dimension of different clusters. 

\todo[inline]{Algoritme}

\todo[inline]{Plaatje}

\todo[inline]{Blaat}

\todo[inline]{Theorie?}	

\subsection{Connectivity}
\label{ss:exp:connectivity}
%!TEX root = practicum2.tex
\begin{figure}[b!]
	\centering
	\begin{subfigure}{0.45\columnwidth}
		\centering
		\includegraphics[width=\textwidth]{img/assignment_connectivity_four_N80_p3.jpeg}
		\caption{4-connectivity}
		\label{fig:exp:connectivity:fourConnect}
	\end{subfigure}
	\begin{subfigure}{0.45\columnwidth}
		\centering
		\includegraphics[width=\textwidth]{img/assignment_connectivity_eight_N80_p3.jpeg}
		\caption{8-connectivity}
		\label{fig:exp:connectivity:eightConnect}
	\end{subfigure}	
	\caption{The results of growing a cluster with the same grid of probabilites with \subref{fig:exp:connectivity:fourMask} four-connectivity and \subref{fig:exp:connectivity:eightMask} eight-connectivity. Note that although the clusters are generated on a grid with $N = 80$, we have plotted them on a grid with $N = 16$.}
	\label{fig:exp:connectivityResults}
\end{figure}

We consider two different connectivities, namely four- and eight-connectivity, which are illustrated in \cref{fig:exp:connectivity}. In this section we discuss the influence of the 8-connectivity on the size of the cluster.

\Cref{fig:exp:connectivityResults} shows two clusters which have been grown using the same probabilities but different connectivities. We see that in this case the 8-connected cluster is much larger than the 4-connected cluster. Although a different seed for the algorithms may result in different clusters in general one would expect the 8-connected cluster to grow larger than the 4-connected version. Since the expected number of sites that are occupied in each grow step when 8-connectivity is used is twice as high as the expected number of occupied sites with 4-connectivity. 

% Influence of probability
The results of performing the same experiment as discussed in \cref{ss:exp:probability} with the eight-connectivity mask, are presented in \cref{fig:experiment:conn:mean_std_clusters} and \ref{fig:experiment:conn:p_inf_ratio}. We have changed the range of $p$ to $p = 0.2, 0.21, \dotsc, 0.6$, since even for $p = 0.3$ $P_\infty$ was close to zero. 

\begin{figure}[b!]
	\centering
	\includegraphics[width=\columnwidth]{./img/assignment_d_p_infinite_ratio_p.pdf}
	\caption{Ratio of percolating clusters, $P_\infty$, as a function of $p = 0.2, 0.21, \dotsc, 0.6$ when eight-connectivity is used. Ratios are calculated over $r_{max} = 200$ runs on a $41 \times 41$ grid.}
	\label{fig:experiment:conn:p_inf_ratio}
\end{figure}

% \begin{figure*}
% 	\centering
% 	\includegraphics[width=\textwidth]{./img/assignment_d_mean_std_p.pdf}
% 	\caption{Mean cluster sizes, represented as points, and standard deviations, indicated by the vertical error bars, as a function of $p = 0.2, 0.21, \dotsc, 0.6$ when eight-connectivity is used. The mean and standard deviation were calculated over $200$ runs on a $41 \times 41$ grid.}
% 	\label{fig:experiment:conn:mean_std_clusters}
% \end{figure*}

It should be noted that for some values of $p$, especially larger values there are no mean cluster sizes, since no finite clusters were found. This indicates that one is much more likely to encounter a percolating cluster with eight-connectivity than with four-connectivity for the same value of $p$. Which fits with our ealier observation that the expected number of newly occupied sites with eight-connectivity is twice as high when compared with a four-connected cluster.

That for higher values of $p$ a percolating cluster is more likely than a finite cluster is confirmed by \cref{fig:experiment:conn:p_inf_ratio}, where we find that only for very low values of $p$ $P_\infty$ is zero, and that $P_\infty$ quickly approaches 1. 

These findings suggest that $p_c$ is much lower when eight-connectivity is used. Based on this, admittedly small experiment, one would guess $p_c$ to be approximately $0.2$ when eight-connectivity is used. More research is needed to find the actual value of $p_c$ for eight-connected clusters.\\





% \todo{Influence of size}
% % To determine the influence of the connectivity on the size we have performed the same experiment as used for the four-connectivity. \todo[inline]{Resultaten}

% \todo{Influence on fractal dimension}
% % We have determined the fractal dimension of the cluster generated using four connectivity for $N = 80$ and $p =0.7$. We have found that \todo[inline]{Resulaten}


	

% \section{Conclusion}
% \label{s:conclusion}
% %!TEX root = practicum1.tex
In our discussion of the Chirikov map we have seen some similarities with the logistic map. Both maps are non-linear i.e. they produce output that is not necessary proportional to the input. 

Both maps are driven by a variable which determines the amount of chaos in the system. Meaning that setting this non-linearity parameter will remove any and all chaotic behaviour, and choosing a low value for this parameter limits the chaos. This effect is illustrated for the Chirikov map in \cref{fig:experiment:fancy_k} where for low $K$ values the $x_n$ and $p_n$ values stay approximately in the same area, when $K$ is small. 

An important difference is that the Chirikov map produces two dimensional output, which also depend on each other, see \eqref{eq:chirikov}, whereas the logistic map only produces a one-dimensional map. 

For the logistic map it can be shown that there is period doubling, although the Chirikov map exhibits behaviour that is reminiscent of period doubling, it does not have period doubling. 

\todo[inline]{And some other things.. period doubling, stable points?}

\printbibliography

\end{document}