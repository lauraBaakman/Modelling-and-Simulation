%!TEX root = practicum1.tex

In our discussion of the Chirikov map we have seen some simularities with the logistic map. As is the logistic map, the Chirikov is non-linear i.e. it produces output not necesary proportional to the input. Both maps are also driven by a variables which determines the amount of chaos in the system, and depend on this variable. Meaning that lower values will result in no chaotic behaviour at all. As shown for the Chirikov map in \cref{fig:experiment:fancy_k} where for low $K$ values the $x_n$ and $p_n$ values stay approximately in the same area, when $K$ is small. An important difference is that the Chirikov map produces two dimensional output, which also depend on each other (see \eqref{eq:chirikov}, where the logistic map only produces one. \todo[inline]{And some other things.. period doubling, stable points?}