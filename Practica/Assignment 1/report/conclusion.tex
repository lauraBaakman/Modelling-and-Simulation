%!TEX root = practicum1.tex

In our discussion of the Chirikov map we have seen some similarities with the logistic map. Both maps are non-linear i.e. they produce output that is not necessary proportional to the input. 

Both maps are driven by a variable which determines the amount of chaos in the system. Meaning that setting this non-linearity parameter will remove any and all chaotic behaviour, and choosing a low value for this parameter limits the chaos. This effect is illustrated for the Chirikov map in \cref{fig:experiment:fancy_k} where for low $K$ values the $x_n$ and $p_n$ values stay approximately in the same area, when $K$ is small. 

An important difference is that the Chirikov map produces two dimensional output, which also depend on each other, see \eqref{eq:chirikov}, whereas the logistic map only produces a one-dimensional map. 

For the logistic map it can be shown that there is period doubling, although the Chirikov map exhibits behaviour that is reminiscent of period doubling, it does not have period doubling. 

\todo[inline]{And some other things.. period doubling, stable points?}