%!TEX root = practicum1.tex
\todo[inline]{Korte intro}

\subsection{Trivial Case}
If we choose $K = 0$ \cref{eq:chirikov} becomes:
\begin{subequations}\label{eq:chirikovK0}
	\begin{align}
		\label{eq:chirikov0:p} p_{n + 1} &= p_n \mod 1,\\
		\label{eq:chirikov0:x} x_{n + 1} &= x_n + p_{n + 1} \mod 1.
	\end{align}
\end{subequations}	


\todo[inline]{Analytisch de orbits bepalen voor K = 0}
\todo[inline]{Plotjes als K = 0}

\begin{figure*}
	\centering
	\foreach \k/\fnk in {0/0, 0.2/2, 0.4/4, 0.8/8, 1/10, 2/20, 3/30, 4/40}{
		\begin{subfigure}{0.24\textwidth}
			\centering
			\includegraphics[width=\textwidth]{./img/assignment_b_fancy_k_\fnk.jpg}
			\caption{$K = \k$}
			\label{fig:experiment:fancy_k:\k}
		\end{subfigure}
	}
	\caption{Full 100 run chirikov maps, for different $K$. For each map 1000 iterations and random initialisation for $x_0$ and $p_0$ were used.}
	\label{fig:experiment:fancy_k}
\end{figure*}


\subsection{Non-Trivial Case}
\todo[inline]{How do the orbits change with increasing K? Do we have scenarios that are similar to period doubling?}
\todo[inline]{Discuss similarities and differences to logistic map}

\subsection{KAM-orbits}
\todo[inline]{Refind $K_C$ in literatur and try to confirm that KAM orbits exist just below but not above $K_c$}