%!TEX root = practicum1.tex
\todo[inline]{Korte intro}

\subsection{Trivial Case}
If we choose $K = 0$ \cref{eq:chirikov} becomes:
\begin{subequations}\label{eq:chirikovK0}
	\begin{align}
		\label{eq:chirikov0:p} p_{n + 1} &= p_n \mod 1,\\
		\label{eq:chirikov0:x} x_{n + 1} &= x_n + p_{n + 1} \mod 1.
	\end{align}
\end{subequations}	
Consequently the values of $p_n$ are constant for $n > 0$, since if $p_0 = 1$, $p_1 = 0$ as $1 \mod 1 = 0$. Consquently without modulo 1, $x_n$ would increase linearly. Due to the modulo $x_n$ becomes periodic. 

\Cref{fig:experiment:K0influenceOfX} shows $x_0$ as a function of $n$ for a fixed value of $p_0$, if we comare the different plots  in this figure we observe that changing $x_0$ only influences that phase of $x_n$, and not the period. 

\begin{figure}
	\centering
	\foreach \x/\actualX in {1/0.1, 3/0.3, 5/0.5}{
		\begin{subfigure}[t]{\columnwidth}
			\includegraphics[width=\textwidth]{./img/assignment_b_K=0p_0=04x_0=0\x.pdf}
			\caption{$x_0 = \actualX$}
			\label{fig:experiment:K0:X:\x}
		\end{subfigure}	
	}	
	\caption{$x_n$ as a function of $n$, for $p_0 = 0.4$ and varying $x$.}
	\label{fig:experiment:K0influenceOfX}
\end{figure}

Fixing $x_0$ and varying $p_0$ results in \cref{fig:experiment:K0influenceOfP}, these plots clearly show that $p_0$ influences the periodicity of the map. These plots also illustrate that $p_0$ results in a higher frequency, for fixed $x_0$. 

\begin{figure}
	\centering
	\foreach \p/\actualP in {1/0.1, 3/0.3, 5/0.5}{
		\begin{subfigure}[t]{\columnwidth}
			\includegraphics[width=\textwidth]{./img/assignment_b_K=0p_0=0\p x_0=04.pdf}
			\caption{$p_0 = \actualP$}
			\label{fig:experiment:K0:P:\p}
		\end{subfigure}	
	}	
	\caption{$x_n$ as a function of $n$, for $x_0 = 0.4$ and varying $p$.}
	\label{fig:experiment:K0influenceOfP}
\end{figure}

\begin{figure*}
	\centering
	\foreach \k/\fnk in {0/0, 0.2/2, 0.4/4, 0.8/8, 1/10, 2/20, 3/30, 4/40}{
		\begin{subfigure}{0.24\textwidth}
			\centering
			\includegraphics[width=\textwidth]{./img/assignment_b_fancy_k_\fnk.jpg}
			\caption{$K = \k$}
			\label{fig:experiment:fancy_k:\k}
		\end{subfigure}
	}
	\caption{Full 100 run chirikov maps, for different $K$. For each map 1000 iterations and random initialisation for $x_0$ and $p_0$ were used.}
	\label{fig:experiment:fancy_k}
\end{figure*}


\subsection{Non-Trivial Case}
\todo[inline]{How do the orbits change with increasing K? Do we have scenarios that are similar to period doubling?}
\todo[inline]{Discuss similarities and differences to logistic map}

\subsection{KAM-orbits}
\todo[inline]{Refind $K_C$ in literatur and try to confirm that KAM orbits exist just below but not above $K_c$}