%!TEX root = practicum1.tex

\begin{figure*}
	\centering
	% \x/\picname in {4/pic1.png,8/pic2.png,15/pic3.png,16/pic4.png}
	\foreach \dim/\x/\p in {0/0.500/0.500, 1/0.1576131/0.9705928, 2/0.1269868/0.9133759}
	{ 
		\begin{subfigure}[t]{0.32\textwidth}
			\includegraphics[width=\textwidth]{./img/assignment_a_\dim_dim.pdf}
			\caption{$x_0=\num{\x}$, $p_0=\num{\p}$}
			\label{fig:experiment:dimension:\dim}
		\end{subfigure}
		\begin{subfigure}[t]{0.32\textwidth}
			\includegraphics[width=\textwidth]{./img/assignment_a_\dim_dim_progression_p.pdf}
			\caption{Progression of $p$ in \subref{fig:experiment:dimension:\dim}}
			\label{fig:experiment:dimension:\dim:x}
		\end{subfigure}		
		\begin{subfigure}[t]{0.32\textwidth}
			\includegraphics[width=\textwidth]{./img/assignment_a_\dim_dim_progression_x.pdf}
			\caption{Progression of $x$ in \subref{fig:experiment:dimension:\dim}}
			\label{fig:experiment:dimension:\dim:p}
		\end{subfigure}		
	}
	\caption{Each row corresponds to on set of initial values $\left\langle x_0, p_0 \right\rangle$. The first column shows $x_n$ versus $p_n$ for $n \in \left[0,\, \num{10000} \right]$. The second and third column depict respectively the progression of $p$ and $x$.}
	\label{fig:experiment:dimension}
\end{figure*}