%!TEX root = practicum1.tex
\todo[inline]{Mogelijk uitbreiding als Han er zo niet gelukkig mee is: bespreek Winding numbers, zie \cite{kenzel1997physics}, en bepsreek de praktische toepassing van de map in de inleiding.}

This paper discusses the Chirikov map, since we are only interested in the non-integer part of the Chirikov map we the following modified map:
\begin{subequations}\label{eq:chirikov}
	\begin{align}
		\label{eq:chirikov:p} p_{n + 1} &= p_n + \frac{K \sin \left(  2 \pi x_n \right)}{2 \pi} \mod 1 \\
		\label{eq:chirikov:x} x_{n + 1} &= x_n + p_{n + 1} \mod 1,
	\end{align}
\end{subequations}	
where the non-linearity parameter $K \in \mathbb{R}$. The modulo operator ensures that $x_n, p_n \in \left[ 0, 1 \right]$. In this paper we discuss the influence of the different parameters on the orbits in the map.