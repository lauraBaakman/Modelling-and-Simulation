%!TEX root = practicum2.tex
\todo[inline]{Beter inleiding, misschien beginnen met bacterie in petri schaaltje}
\noindent \textcite{kenzel1997physics} give the following interpretation of the percolation model; it describes the geometry of the randomly generated pores in a porous material through which only certain particles can percolate if the pores form continuous paths. We model this material using a finite lattice, although different lattices are possible, we consider only a square lattice. 

The exact percolation model is describes in \cref{s:method}, this section also presents an implementation of the model in pseudo code. In \cref{s:experiment} we discuss some of the experiments we have performed with the model and their results. 
% \Cref{s:conclusion} presents a summary of the findings of our experiment. 
